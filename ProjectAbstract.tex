\documentclass[12pt, a4paper]{article}


\usepackage{fixltx2e}
\usepackage{booktabs}
\usepackage{hyperref}

\usepackage[final, activate=true,protrusion=true, kerning=true, expansion=true ]{microtype}

\author{
Saurabh Mathur \\
\texttt{14BIT0180}
\and
Abishek Aditya \\
\texttt{14BIT0137}
\and
Tushar Bhatia \\
\texttt{14BIT0163}
}

\title{File Compressor}

\begin{document}

    \maketitle
    
    
    \section*{Problem Statement}
    This project deals with the problem of storage and transmission of large files. \\
    We aim to create an application that performs data compression using the Lempel Ziv Welch (LZW) universal lossless data compression algorithm. 
    \\ This will facilitate efficient storage and transmission.
    \section*{Input}
    The input will consist of a setting that is, encode or decode, the file-name for an input file and the name of the output file (optional). The input file may be of any type.
    \section*{Concepts Used}
    Following Concepts are used -
        \begin{itemize}
        \item Object Oriented Programming
            \begin{itemize}
                \item Encapsulation
                \item Abstraction
                \item Inheritance
                \item Polymorphism.
        
        \end{itemize}
        \item Data Structures
            \begin{itemize}
                \item Hash Map or Dictionary
            \end{itemize}
        \item Input/Output
            \begin{itemize}
                \item File Streams
                \item Command Line Arguments
            \end{itemize}
        \item Error Handling and Exceptions
        \end{itemize}
    \section*{Methodology}
    There will be three major components - the encoder, the decoder and the input/output system. \\The input/output system will parse the command line arguments, read the input file and suitably call the encode or decode functions with its contents. \\If there is time before the deadline, a Graphical User Interface (GUI) shall also be implemented.  \\
    The source code will be updated time to time on the Project Repository at \texttt{\small{https://github.com/saurabhmathur96/ITE201-File-Compressor}}.
    \section*{Work Distribution}
        \begin{tabular}{llr}
        \toprule
        \cmidrule(r){1-2}
        Person    & Component \\
        \midrule
        Abishek      & Decoder  \\
        Saurabh   & Encoder       \\
        Tushar       & File-Input/Output     \\
        \bottomrule
        \end{tabular}
    \section*{Output}
    The output will consist of a single output file that will have been encoded or decoded depending on the input.
    \section*{Timeline}
    The project shall be completed by May 8, 2015.
    \section*{Sources of Information}
    \url{http://www.cs.cf.ac.uk/Dave/Multimedia/node214.html}\\
    \url{http://en.wikipedia.org/wiki/Lempel%E2%80%93Ziv%E2%80%93Welc}
\end{document}
